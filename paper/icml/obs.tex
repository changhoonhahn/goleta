\section{Observations: NASA-Sloan Atlas}  \label{sec:nsa}
In this work, we analyze observed galaxy photometry from the NASA-Sloan
Atlas\footnote{\url{http://www.nsatlas.org/}} (hereafter NSA).
The NSA provides photometry of $z < 0.05$ galaxies observed by the Sloan
Digital Sky Survey~\citep[SDSS;][]{aihara2011} Data Release 8 with improved
background subtraction~\citep{blanton2011}. 
We use optical $g$, $r$, $i$, $z$ band absolute magnitudes derived using 
{\sc kcorrect}~\citep{blanton2007}, assuming a
\cite{chabrier2003} initial mass function (IMF). 
%In Fig.~\ref{fig:nsa}, we present the $(g - r) - M_r$ color-magnitude distribution of $\sim$120,000 NSA galaxies (black).

Out of the full NSA sample, we focus on luminous galaxies with 
$-18 > M_r > -22$.
%We exclude galaxies brighter than $M_r > -22$ since our simulated galaxy sample does not include a large number of the most luminous galaxies due to cosmic variance. 
In addition, we only select galaxies with precisely measured photometry: 
magnitude uncertainties below $\sigma_g, \sigma_r, \sigma_i < 0.022$ and 
$\sigma_z < 0.04$. 
Lastly, we impose the color cuts to exclude galaxies outside the central 68
percentile of the $g-r$, $g-i$, $g-z$, $r-i$, $r-z$, $i-z$ color
distributions.
The color cuts remove NSA galaxies that potentially have observational
artifacts or problematic photometry.
They also ensure that the NSA galaxies are within the photometric distribution
(\ie~support) of our simulated galaxies.
We mark the 95$^{th}$ percentile contour of our NSA subsample in
Fig.~\ref{fig:nsa} (black dot-dashed).
In total, we use 14,736 NSA galaxies.


