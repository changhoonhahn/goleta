\section{Introduction} \label{sec:intro} 
In recent work, Villaescusa-Navarro~et~al.~\yrcite{villaescusa-navarro2022}
showed that it is possible to place cosmological constraints from only the
internal properties of a single galaxy.
They used galaxies from 2,000 state-of-the-art hydrodynamical simulations with
different cosmologies and astrophysical models from
CAMELS~\citep{villaescusa-navarro2021, villaescusa-navarro2022a} to train
moment networks~\citep{jeffrey2020a} that predict cosmological parameters from
galaxy properties. 
With only a handful of galaxy properties, including stellar mass ($M_*$),
stellar metallicity ($Z_*$), and maximum circular velocity ($V_{\rm max}$),
they were able to constrain $\Omega_m$ to 10\% precision with a single galaxy.
They found similar constraining power for galaxies simulated using the subgrid
physics models of the IllustrisTNG~\citep{pillepich2018, weinberger2018} and
SIMBA~\citep{dave2019}. 
Since then, follow-up works have found consistent results for other
hydrodynamicl models: \citep{echeverri2023}. 


According to Villaescusa-Navarro~et~al.~\yrcite{villaescusa-navarro2021}, the
cosmological information is derived from the imprint of $\Omega_m$ on the dark
matter content of galaxies that affects galaxy properties in a distinct way
than astrophysical processes.
Also, since $\Omega_b$ is fixed in CAMELS, which is justified by the tight
constraints from Big Bang Nucleosynthesis, the galaxy properties are
effectively measuring the baryon fraction, $\Omega_b/\Omega_m$.
For instance, $V_{\rm max}$ measures the depth of the total matter
gravitational potential while other properties like $M_*$ and $Z_*$ measures
the mass in baryons so together they can constrain the ratio
$\Omega_b/\Omega_m$.
In fact, a similar approach was used three deacdes ago in 
White et al. \yrcite{white1993} to constrain $\Omega_b/\Omega_m$  using galaxy
clusters. 

Despite the promising signs that they may be useful cosmological probes, galaxy
properties themselves are {\em not} actual observable.
They are derived quantities that are typically inferred from photometry or
spectra of galaxies and require modeling the spectral energy distribution
(SED) or emission lines~\citep{conroy2013}.
In this work, we go beyond the theoretical considerations of previous works and
infer cosmological parameters from actual galaxy observables --- optical  
photometry.  
We leverage a simulation-based inference method that employs neural density
estimation to estimate the posterior of cosmological parameters given galaxy
photometry, similar to the approach of \cite{hahn2022a}. 
Furthermore, since we expect a limited amount of cosmological information from
the photometry of a single galaxy, we present a hierarhical population
inference approach for inferring the posterior of cosmological parameters
from the photometry of multiple galaxies. 
Lastly, we present the cosmological constraints derived from applying this
approach to the photometry of $\sim$15,000 SDSS galaxies from the NASA-Sloan
Atlas (Sec.~\ref{sec:nsa}). 
